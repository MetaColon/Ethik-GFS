\documentclass{article}

\usepackage[utf8]{inputenc}
\usepackage[german]{babel}
\usepackage[
backend=biber,
style=reading
]{biblatex}

\addbibresource{resource.bib}

\title{Überwachung \\ Ethik Ausarbeitung}
\author{Timo Borner}
\date{2018}

\sloppy

\begin{document}
\maketitle
\newpage
\tableofcontents
\newpage

\section{Einleitung}
Das Thema der Überwachung spielt schon seit über einem Jahrhundert eine wichtige Rolle in unserer Gesellschaft. Zum Beispiel wurde im ersten Weltkrieg viel Kritik durch Überwachung ausgelöst \autocite{Pressefreiheit}.
Hier wurde die Überwachungen durch verschiedene Zensuren, wie z.B. der Briefzensur \autocite{Zensur} umgesetzt. Doch nicht nur vor über 100 Jahren wurde Überwachung kritisiert: Auch heute beschweren sich viele Menschen darüber, dass nicht genug Acht auf Datenschutz gelegt werde, dass Superkonzerne wie Google alle Daten bekämen oder dass uns eine Welt bevorstehe, in der es nicht nur keine Privatsphäre, sondern auch keine Freiheiten mehr gebe. Und dennoch, wenn man vergleicht, wie die Überwachung vor Hundert Jahren aussah und wie sie heute aussieht, kommt man nicht umhin, festzustellen, dass wir immer mehr überwacht werden. Vielleicht ist die Überwachung, wenn nicht gar die totale Überwachung, in Wirklichkeit gar nichts Schlechtes, sondern sogar etwas Notwendiges?

\section{Arten von Überwachung}
Das Thema der Überwachung könnte man in zwei Arten einteilen: Die Totale und die Partielle Überwachung. Ich möchte anmerken, dass das keineswegs eine offizielle Unterteilung ist, ich werde sie aber fortan verwenden.

\subsection{Partielle Überwachung}
Die Partielle Überwachung ist die Art der Überwachung, die wir derzeit und das vergangene Jahr über erlebt haben. Wie der Name schon sagt (Partiell = nicht vollständig ausgeprägt \autocite{Partiell}), findet eine Überwachung nur teilweise statt. Z.B. werden wir heutzutage im Internet überwacht (sofern man sich nicht anderweitig schützt - s. Kapitel \ref{sec:Schutz vor Ueberwachung}) oder in England wird man im öffentlichen Raum mit Überwachungskameras ("CCTV") überwacht. Es besteht also eine Überwachung - aber diese ist nicht vollständig. Es gibt immer noch Bereiche im Leben, in denen man nicht überwacht wird (wie z.B. das eigene Wohnzimmer). Falls man in diesen Bereichen dennoch überwacht wird, geschieht es auf freiwilliger Basis (wenn man sich z.B. eine "Alexa" in sein Wohnzimmer stellt, geschieht das nicht aus Pflicht).

\subsection{Totale Überwachung}
Die Totale Überwachung hingegen kennt keine Grenzen und ist absolut. Herrschte eine Totale Überwachung, gäbe es keine Bereiche in unseren Leben, die nicht überwacht würden. Jede Aktion, jede Meinungsäußerung, würde von einer oder mehreren Institutionen bemerkt und analysiert werden. Diese Art der Überwachung wird in Distopien, wie z.B. Orwells 1984 - s. Kapitel \ref{subsec:1984}) beschrieben, in der Realität existiert sie aber (noch) nicht.

\section{Praktische Anwendung}
Die Partielle Überwachung wird und wurde bereits mehrfach angewendet.

\subsection{Drittes Reich\autocite{DrittesReich}}
Im Dritten Reich wurde die Überwachung vorwiegend von dem SD ausgeübt. SD steht für Sicherheitsdienst und war eine Abteilung der SS (Schutzstaffel). Der SD bestand aus sogenannten V-Männern. Das waren scheinbar normale Leute, die ihr soziales Umfeld ausspionierten und den Nationalsozialisten Bericht erstatteten. Interessant ist hier, dass die Überwachung durch die Bürger selbst ausgeführt aber durch die Regierung analysiert wurde. Diese Überwachung hat es im Dritten Reich überhaupt erst möglich gemacht, dass politische Gegner in KZs kamen. Die Überwachung hat hier den Terror also begünstigt.

\subsection[England]{England\autocite{England}}
England ist das Land in Europa mit den meisten Überwachungskameras. Es sind über 6.000.000 Kameras in ganz England installiert und es werden immer mehr. Der Sinn der Kameras soll der Schutz vor Verbrechen und Terror sein, allerdings wird nur ca. 0,1\% der Fälle mithilfe von Überwachungskameras gelöst. Folglich wird die ausgebaute Überwachung in England auch stark kritisiert. Vor allem ein neuerer Vorfall Ende letzten Jahres sorgte für viel Kritik, als eine Schule Überwachungskameras auf Schultoiletten installierte \autocite{Toilette}.

\subsection{Internet\autocite{Internet}}
Schon seit längerem wird die mangelnde Privatsphäre und die Überwachung im Internet kritisiert. Das CIA überwacht die Nutzer des Internets legal, "Hacker" machen es illegal. Die Überwachung im Internet scheint jedoch trotz der Kritik, die sie erfährt, nur noch zu wachsen: Erst letztes Jahr wurde es offiziell erlaubt, dass verschlüsselte Chatverläufe mithilfe eines sogenannten "Staatstrojaners" von der Regierung mitgelesen werden darf. Es gibt jedoch eine Gegenbewegung dazu, die sich interessanterweise ebenfalls Hacker nennen, die trotz aller Überwachungen im Internet sich dennoch bemühen, anonym zu surfen (s. Kapitel \ref{sec:Schutz vor Ueberwachung}).

\section{Werke}
Es gibt verschiedene Werke, welche sich mit dem Thema der Überwachung auseinandersetzen - ich möchte hier zwei ansprechen.

\subsection{1984\autocite{1984}}
\label{subsec:1984}
In dem Roman "1984" geht es um einen totalitären Überwachungsstaat. Hauptperson des Buches ist Winston Smith. Dieser versucht gegen das System vorzugehen, scheitert allerdings letztlich und ergibt sich dem System.
Das System ist so umgesetzt, dass es verschiedene Ministerien mit verschiedenen Aufgaben gibt. Diese Ministerien verkörpern genau das Gegenteil, von dem, was ihr Name verspricht. So gibt es das Miniwahr (das Ministerium für Wahrheit), das alles so verdreht und entsprechende Lügen verbreitet, dass das System überlebt. Es gibt das Minipax (Ministerium für Frieden), dass dafür verantwortlich ist, dass sich der Staat dauerhaft im Krieg mit einem der zwei anderen Staaten befindet, das Minifülle (Minsterium für Überfluss), das die Nahrung rationiert und das Minilieb (Ministerium für Liebe), das für rechtliche Angelegenheiten zuständig ist. Außerdem gibt es in 1984 eine sogenannte "Gedankenpolizei", die darauf aufpasst, dass keiner eine Meinung hat, die dem System schaden könnte. So wird Smiths Widerstand letztlich auch von der Gedankenpolizei aufgelöst. Es erschien so, als sei er zusammen mit einigen anderen im Widerstand gewesen, letztlich waren die anderen Mitstreiter aber von der Gedankenpolizei und hatten ihn verhaftet.

In 1984 ist es praktisch unmöglich, eine andere Meinung zu haben als die Machthabenden. Sobald man einer anderen Meinung ist (ob man diese äußert oder nicht - relevant ist nur, dass die Gedankenpolizei davon erfährt) wird man so lange gefoltert, bis man letztlich die Meinung der Machthabenden akzeptiert: Das passiert auch Smith, der letztlich die Lügen von 1984 glaubt.

Es ist zweifelsfrei erkennbar, dass es sich hier um eine Distopie handelt - keiner möchte in einer Welt leben, in der es absolut keine Freiheiten gibt. Und diese Welt wird hier durch die totale Überwachung ermöglicht: Gäbe es keine totale Überwachung, könnte die Gedankenpolizei nicht erkennen, ob Menschen sich gegen das System stellen.

\subsection{The Circle\autocite{TheCircle}}
In dem Roman "The Circle" geht es darum, wie sich nach und nach ein System totaler Überwachung aufbaut. Der "Circle" ist ein Unternehmen, welches sehr fortschrittliche Technologien verwendet und schon am Anfang des Buches extrem beliebt ist. Die Hauptperson Mae bekommt die Chance, ein Mitarbeiter des Circles zu werden und steigt dort relativ schnell auf. So erlebt der Leser mit, wie nach und nach Technologien veröffentlicht werden, die eine totale Überwachung ermöglichen, wie z.B. winzige Kameras, welche man überall anbringen kann. Interessant ist, dass Mae dieses System die meiste Zeit zu akzeptieren scheint und der Leser somit dazu gebracht wird, zu glauben, dass es tatsächlich gut ist. Letztendlich endet das Buch aber doch distopisch, als die letzte Chance, die totale Überwachung zu verhindern nicht von Mae ergriffen wird und sie selbst nach einem Zeitsprung sogar die Überwachung der Gedanken vorschlägt. Interessant ist ebenfalls, dass, nicht wie in "1984", die Überwachung nicht durch den Staat und auch nicht durch den Circle ausgeübt wird, sondern von den Menschen selbst, vom Kollektiv. Denn jeder hat Zugriff auf die Bilder der Kameras und andere Daten des Circles und sobald sich jemand der generellen Idee des Circles widersprüchlich verhält, bekommt das Kollektiv, welches den Circle generell befürwortet, das mit.

Auch hier ist die Kritik an totaler Überwachung, auch oder gerade wenn sie durch das Kollektiv ausgeübt wird, stark erkennbar, auch wenn auch einige positive Aspekte dargestellt werden: Z.B. soll (und wird) mit der Überwachung die Vergewaltigung von Kindern verhindert (werden).

\section{Persönliche Abwägung}
Mehr Überwachung bringt zweifelsohne Vorteile: Wenn sich Menschen überwacht fühlen, begehen sie nicht so schnell ein Verbrechen, was die generelle Sicherheit der Menschen fordert - und wie sonst bringt man Menschen dazu, sich überwacht zu fühlen, wenn nicht durch Überwachung? Es kann nicht abgestritten werden, dass mehr Sicherheit im Allgemeinen erstrebenswert ist. Wenn nun also mehr Überwachung mehr Sicherheit zur Folge hat, wäre doch eine totale Überwachung und somit maximale Sicherheit ein erstrebenswerter Zustand.

Allerdings hat mehr Überwachung nicht nur den Vorteil von mehr Sicherheit, sondern auch einige Nachteile. Die totale Überwachung hat zum Beispiel den Nachteil, dass es praktisch keine Meinungsfreiheit und generell keine wirkliche Freiheit mehr geben würde. Denn wenn jeder jederzeit überwacht wird, wird er immer so handeln, wie er glaubt, dass es der Überwachende bzw. die Überwachenden für richtig empfinden würden und würde keine Handlungen aus einem freien Willen heraus unternehmen. Außerdem wird vor allem durch die totale Überwachung durch das Kollektiv die Meinung von Minderheiten missachtet. Denn wenn nur die Meinung des Kollektivs zählt, zählt die am meisten vertretene Meinung, ganz egal, ob es Minderheiten mit anderen Meinungen gibt. Totale Überwachung ist also nicht anzustreben, aber was spricht gegen partielle Überwachung? Hier ist die sogenannte "Slippery Slope" ein Problem. Denn wenn man anfängt zu überwachen, fällt es einem leichter, mehr zu überwachen. So war es z.B. in England: Am Anfang gab es wenige Kameras in wichtigen Bereichen, dann folgten immer mehr Kameras überall verteilt. Eine partielle Überwachung führt also leicht zu einer totalen Überwachung.
Letztlich ist noch das Prinzip "Wissen ist Macht" anzuführen, was vor allem bei der Überwachung nicht durch den Staat sondern durch Unternehmen wie z.B. Google anzuführen ist. Denn, ähnlich wie der "Circle", sammelt Google bereits jetzt enorme Mengen an Daten über die Nutzer. Diese Daten geben Google oder ähnlichen Firmen eine enorme Macht und man könnte sich fragen, ob man einem Konzern wirklich so viel Macht zugestehen möchte.

Letztlich ist es nicht objektiv bewertbar, ob der Vorteil der erhöhten Sicherheit die Nachteile überwiegt, doch für mich persönlich wird meine Freiheit immer wichtiger als meine Sicherheit sein.

\section{Schutz vor Überwachung}
\label{sec:Schutz vor Ueberwachung}
Was kann man also tun, wenn man nicht überwacht werden möchte? Diese Frage muss individuell auf die verschiedenen Bereiche angewendet werden.

Vor öffentlichen Überwachungskameras kann man sich praktisch nicht schützen, außer man meidet überwachte Bereiche oder "tarnt" sich, was aber kaum zu empfehlen wäre, weil es für gewöhnlich mehr Aufmerksamkeit erregt.

Der Schutz vor Überwachung ist vor allem im Bereich des Internets interessant. Menschen, die ohne Überwachung ("anonym") im Internet unterwegs sein möchten, nennen sich häufig "Hacker". Die Internet Anonymität ist eine Wissenschaft für sich und ich werde ihr hier nicht gerecht werden - für weitere Informationen ist die Seite \textit{https://www.csoonline.com/article/2975193/data-protection/9-steps-completely-anonymous-online.html} zu empfehlen. Zumindest zu erwähnen ist jedoch die wohlbekannte Technik, über das sogenannte "Tor"-Netzwerk auf das Internet zuzugreifen. Dies kann mithilfe eines Tor-Browsers realisiert werden und funktioniert vereinfacht so, dass jede Anfrage über eine Reihe anderer Computer ("Nodes") läuft, sodass nicht nachvollzogen werden kann, wer die Anfrage gesendet hat.
Eine weitere Technik ist es, ein VPN ("Virtual Private Network") zu nutzen, welches im Grunde jede Anfrage verschlüsselt und über ein anderes Netzwerk verschickt, wodurch jene Anfrage auch wieder nicht zurückverfolgt werden kann.

\section{Fazit}
Meiner Meinung nach ist Überwachung ein gefährliches Thema und vor allem im Internet, wo die Überwachung vorwiegend von Mega-Konzernen wie Google durchgeführt wird, ist es zu empfehlen, auf Anonymität zu achten.

\newpage
\printbibliography
\end{document}
